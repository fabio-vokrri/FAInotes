\section{Introduction}
Artificial intelligence is a discipline that spans across science and engineering:
\begin{itemize}
    \item Science tries to understand what intelligence is.
    \item Engineering tries to implement an intelligent agent.
\end{itemize}
In some sense it tries to model the process of \textbf{conscious reasoning} and \textbf{problem solving}, taking inspiration from some of the structures that occur in the brain.

There isn't a single definition of intelligence: some define intelligence in terms of fidelity to human performance, while others prefer a formal definition of intelligence called \textbf{rationality}. Some consider intelligence to be a property of reasoning and thought processes, while others focus on intelligent behavior.  

From these two characterizations (human v. rational/ thought v. behavior) there are four possible combinations that cross several scientific disciplines.

\subsection{Acting Humanly - Turing Test Approach}
Alan Turing proposed the Turing test, a mental experiment to answer the question whether or not a computer can reach human-like performance. The test is passed if a human interrogator, after posing some written questions, cannot tell if the written responses came from a person or a computer. In this case we are not interested to know how the agent "thinks", but more on behaving as a human.

To pass the test, the computer would need the following capabilities:
\begin{itemize}
    \item Natural language;
    \item Knowledge representation;
    \item Automated reasoning;
    \item Machine learning.
\end{itemize}

\subsection{Thinking Humanly - Cognitive Modeling Approach}
In order to say that a program thinks like a human, we must know how humans think. We can learn about human thoughts via an approach called \textbf{cognitive modeling} in three main ways:
\begin{enumerate}
    \item Introspection: trying to capture our own thoughts;
    \item Psychological experiments: observing human behavior;
    \item Brain imaging: observing the brain in action.
\end{enumerate}

Once we have a precise theory of the mind, it can be said that if a program's input-output behavior matches a corresponding human behavior, that is evidence that some of the program's mechanisms could also be present in a human. 

\subsection{Thinking Rationally - "Laws of thought" Approach}
This approach is based on the laws of thought, which provide a way to draw the correct conclusions given the right premises. The field that studies these laws is called \textbf{logic} and requires the knowledge of the world that is certain,  a condition rarely met in reality. The theory of probability fills the gap, allowing rigorous reasoning with uncertain information.

\subsection{Acting Rationally - Rational Agent Approach}
An agent is something that autonomously acts, perceives the environment, persists over a prolonged period of time, adapts to the changes, and pursues a goal. A rational agent is one who acts to achieve the best outcome.
This approach of AI has prevailed throughout the field's history because it is more general and amenable to scientific approach than others. AI has focused on the study of agents that do the right thing.